% !TEX root = cv.tex
\documentclass[a4paper]{res}
\usepackage[T1]{fontenc}
\usepackage[T1]{polski}
\usepackage[utf8x]{inputenc}
\usepackage{hyperref}
\usepackage{helvet}
\usepackage[super]{nth}
\usepackage{fancyhdr}
\usepackage{atbegshi,picture}
\AtBeginShipout{\AtBeginShipoutUpperLeft{%
  \put(\dimexpr\paperwidth-1.2in\relax,-1cm)
  {\makebox[0pt][r]{styczeń 2021}}
  \put(\dimexpr\paperwidth-1.2in\relax,-1.5cm)
  {\makebox[0pt][r]{oferta: PCSS — Operator Centrum Zarządzania}}%
}}
\AtBeginShipout{\AtBeginShipoutUpperLeft{%
  \put(\dimexpr1.6in\relax,-1cm)
  {\makebox[0pt][r]{Curriculum Vit\ae}}%
}}
\hoffset=0cm
\usepackage[inner=2.5cm,marginparwidth=4cm,marginparsep=0cm,top=2.24cm,bottom=.2cm]{geometry}
\usepackage[inline]{enumitem}   
\makeatletter
% This command ignores the optional argument for itemize and enumerate lists
\newcommand{\inlineitem}[1][]{%
\ifnum\enit@type=\tw@
    {\descriptionlabel{#1}}
  \hspace{\labelsep}%
\else
  \ifnum\enit@type=\z@
       \refstepcounter{\@listctr}\fi
    \quad\@itemlabel\hspace{\labelsep}%
\fi}
\makeatother
\renewcommand{\labelitemii}{→}
\renewcommand{\labelitemiii}{•}
% \pagestyle{plain}
% \fancyhead[L]{Curriculum Vitae}
% \fancyhead[R]{Q1 2019}
% \lhead{Curriculum Vitae}
\begin{document}
\name{Michał Krzysztof ``Mika'' Feiler}
\address{\bf{Dane osobiste}\\
    \textit{Adres zamieszkania}: os. Jana III Sobieskiego 26D/144,
    60-690 Poznań\\
    \bf{Dane kontaktowe}\\
    \textit{Adres email}: \href{mailto:m@mikf.pl}
{{m@mikf.pl}}\\
	\textit{Numer telefonu}: \href{tel:+48690463069}{+48 690 463 069}\\
}
\address{\bf{Profile internetowe}\\
\href{https://mikf.pl}{{{www.}\textbf{mikf.pl}}}\\
\href{https://linkedin.com/in/mikf}
{\itshape{linkedin.com}\slshape{/in/}\upshape\textbf{mikf}}\\
\href{https://github.com/mkf}
{\itshape{github.com}\slshape{/}\upshape\textbf{mkf}}\\
}
\begin{resume}
    \section{Doświadczenie zawodowe}
    \vspace{-0.05in}
    \begin{tabbing}
        \hspace{2.3in}\= \hspace{2.6in}\= \kill
        \textit{staż:} \textsl{programista}\>
        \textbf{DLabs} \>
        lipiec-sierpień 2016\\
        \> \textsl{Toruń, kujawsko-pomorskie}
        \> czerwiec 2017
    \end{tabbing}\vspace{0pt}
    \begin{itemize}
    \item automatyzacja zarządzania terminalami LTSP w fabryce
    \item automatyzacja konfiguracji bondingu w UCarp za pomocą
    skryptów Bash
    \item automatyzacja wieloinstancyjnej jednoczesnej/inkrementalnej konfiguracji replikacji MongoDB z arbitrem i bez za pomocą skryptów Bash
    \item użycie Dockera i docker-compose
    \item użycie Vagranta z VirtualBox do tworzenia wirtualnych sieciowych środowisk testowych
    \end{itemize}
    \vspace{0in}
    \begin{tabbing}
        \hspace{2.3in}\= \hspace{2.6in}\= \kill
        \textit{staż:} \textsl{programista}\>
        \textbf{kapware.com} \>
        lipiec-sierpień 2017\\
        \> \textsl{Toruń, kujawsko-pomorskie}
    \end{tabbing}
    \begin{itemize}
    \item wrappery API i elementy UI do aplikacji w ClojureScript/ReactNative na Androida
    \end{itemize}
    \section{Wykształcenie}
    \vspace{-0.05in}
    \begin{tabbing}
        \hspace{1in}\= \kill
        \textit{obecnie} \>
        \textbf{Uniwersytet Adama Mickiewicza w Poznaniu}\textit{,
        Wydział Matematyki i Informatyki}
    \end{tabbing}\vspace{-20pt}
    II rok\hspace{0.5in}{\textbf{Informatyka}, studia stacjonarne I stopnia
    (inżynierskie) 3.5-letnie}
    \section{Languages}
    \vspace{0.05in}
    \textbf{Angielski} — \textsl{średnio zaawansowany}; ~ ~ 
    Francuski — \textsl{podstawowo}; ~ ~ ~
    \textit{Polski} — \textsl{natywnie}
    \section{Umiejętności}
    \vspace{0.1in}
    \begin{itemize}
    		\item zaznajomienie z językami Go, Java, Python, Bash
    		\item elementarne z C, C++, JavaScript, PHP, Perl5, Ruby, Dart, Pascal
    		\item zainteresowanie językami Rust, Nim, Kotlin, Ocaml
       	\item pojęcie o paradygmatach programowania funkcyjnym i reaktywno-funkcyjnym, w tym podstawy języków funkcyjnych Haskell i Clojure
        \item zaznajomienie z systemami GNU/Linux (Ubuntu, Debian, Red Hat jako Fedora i CentOS, Arch Linux, NixOS), a także FreeBSD i OpenBSD
        \item znajomość systemów kontroli wersji Git, Mercurial
        \item zainteresowanie Test Driven Development
        \vspace{0.04in}
        \item doświadczenie w pracy w metodologiach zwinnych (\textsl{agile}), znajomość Scruma
        \item zaznajomienie z metodyką Porozumienia Bez Przemocy (\textsl{Nonviolent Communication, NVC})
    \end{itemize}
    \section{Aktywność pozazawodowa}
    \begin{itemize}
        \item regularny uczestnik meetupów IT i Agile między 2016 a 2019
        \item aktywny członek Stowarzyszenia Mensa Polska od 2014
        \item od 2016 do 2018 członek Koła Naukowego Informatyków
        Uniwersytetu Mikołaja Kopernika w Toruniu: m.in.
        współprowadzenie warsztatów z systemu kontroli wersji Git
        oraz pomoc organizacyjna przy warsztatach z Clojure		
    \end{itemize}
\end{resume}
\begin{center}
    \noindent\makebox[\dimexpr\linewidth]{\rule{\dimexpr\paperwidth-3in}{0.4pt}}
    \footnotesize
    Zgadzam się i wyrażam zgodę bla bla bla tu wstawiać później stopkę wedle życzenia pracodawcy.\\
    A do publikacji na stronie wyrzucić to stąd.
\end{center}
\end{document}