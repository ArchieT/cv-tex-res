% !TEX root = cv.tex
\documentclass[a4paper]{res}
\usepackage[T1]{fontenc}
\usepackage[T1]{polski}
\usepackage[utf8x]{inputenc}
\usepackage{hyperref}
\usepackage{helvet}
\usepackage[super]{nth}
\usepackage{fancyhdr}
\usepackage{atbegshi,picture}
\AtBeginShipout{\AtBeginShipoutUpperLeft{%
  \put(\dimexpr\paperwidth-1.2in\relax,-1cm)
  {\makebox[0pt][r]{Q1 2020}}%
}}
\AtBeginShipout{\AtBeginShipoutUpperLeft{%
  \put(\dimexpr1.6in\relax,-1cm)
  {\makebox[0pt][r]{Curriculum Vit\ae}}%
}}
\hoffset=0cm
\usepackage[inner=2.5cm,marginparwidth=4cm,marginparsep=-0.5cm,top=1cm,bottom=.2cm]{geometry}
\usepackage[inline]{enumitem}   
\makeatletter
% This command ignores the optional argument for itemize and enumerate lists
\newcommand{\inlineitem}[1][]{%
\ifnum\enit@type=\tw@
    {\descriptionlabel{#1}}
  \hspace{\labelsep}%
\else
  \ifnum\enit@type=\z@
       \refstepcounter{\@listctr}\fi
    \quad\@itemlabel\hspace{\labelsep}%
\fi}
\makeatother
\renewcommand{\labelitemii}{→}
\renewcommand{\labelitemiii}{•}
% \pagestyle{plain}
% \fancyhead[L]{Curriculum Vitae}
% \fancyhead[R]{Q1 2019}
% \lhead{Curriculum Vitae}
\begin{document}
\name{Michał Krzysztof Feiler}
\address{\bf{Address of residence}\\
    os. Jana III Sobieskiego 26D/144 \\
    60-690 Poznań, wielkopolskie, Poland
}
\address{\bf{Contact information}\\
\textit{Email}: ~ \href{mailto:m@mikf.pl}
{{m@mikf.pl}}\\
\textit{Website}: \href{https://mikf.pl}
{{{www.}\textbf{mikf.pl}}}\\
\textit{Mobile}: \href{tel:+48605623668}{+48 605 623 668}\\
\href{https://linkedin.com/in/mikf}
{\itshape{linkedin.com}\slshape{/in/}\upshape{mikf}}\\
}
\begin{resume}
    \vspace{-0.13in}
    \section{Education}
    \vspace{-0.05in}
    \begin{tabbing}
        \hspace{1.3in}\= \kill
        2014 – 2017 \>
        IV Liceum Ogólnokształcące
        im. Tadeusza Kościuszki w Toruniu
    \end{tabbing}
    \vspace{-10pt}
        \marginpar{\slshape\footnotesize/high school/}
    \vspace{-10pt}
    \hspace{1in}\textsl{na profilu matematyczno-fizyczno-informatycznym (klasa ,,uniwersytecka'')}
    \vspace{-0.2in}
    \begin{tabbing}
        \hspace{1.3in}\= \kill
        2017~/~2018 \>
        \textsl{Nicolaus Copernicus University in Toruń}\textit{,
        Faculty of Mathematics and Computer Science}
    \end{tabbing}\vspace{-20pt}
    \hspace{1in}\textsl{Informatics} (unfinished \nth{1} year)
    \vspace{-0.2in}
    \begin{tabbing}
        \hspace{1.3in}\= \kill
        2018~/~2019 \>
        \textsl{Adam Mickiewicz University in Poznań}\textit{,
        Faculty of Mathematics and Computer Science}
    \end{tabbing}\vspace{-20pt}
    \hspace{1in}{\textsl{Informatics} (unfinished \nth{1} year)}
    \vspace{-0.2in}
    \begin{tabbing}
        \hspace{1.3in}\= \kill
        2019 – \sl{2023} \>
        \textbf{Adam Mickiewicz University in Poznań}\textit{,
        Faculty of Mathematics and Computer Science}
    \end{tabbing}\vspace{-20pt}
    \hspace{1in}{\textbf{Informatics}}
    \vspace{-0.2in}
    \section{Experience}
    \vspace{-0.05in}
    \begin{tabbing}
        \hspace{2.3in}\= \hspace{2.6in}\= \kill
        \textit{intership:} \textsl{programmer}\>
        \textbf{DLabs} \>
        July-August 2016\\
        \> \textsl{Toruń, kujawsko-pomorskie}
        \> June 2017
    \end{tabbing}\vspace{-20pt}
    \hspace{.5in}\textsl{LTSP thin\&fat terminals management automatization in a factory; Python, Docker}
    \vspace{-0.17in}
    \begin{tabbing}
        \hspace{2.3in}\= \hspace{2.6in}\= \kill
        \textit{intership:} \textsl{programmer}\>
        \textbf{kapware.com} \>
        July-August 2017\\
        \> \textsl{Toruń, kujawsko-pomorskie}
    \end{tabbing}\vspace{-17pt}
    \hspace{.5in}\textsl{API wrappers, UI elements for ClojureScript/ReactNative (Android) apps}
    \vspace{-0.17in}
    \section{Languages}
    \vspace{0.05in}
    \textbf{English} — \textsl{intermediate}; ~ ~ 
    French — \textsl{elementary}; ~ ~ ~
    \textit{Polish} — \textsl{native}
    \section{Skills}
    \vspace{0.1in}
    \begin{itemize}
        \item good grasp of functional and reactive-functional programming paradigms
        \item \textsc{Programming languages} \hspace{0.2in}
        \footnotesize{\textsl{
            (I know LoC count is not a good measure
            of programmer's experience)}}
            \normalsize
        \vspace{-0.02in}
        \begin{itemize}
            \item over the last few years did tens of thousands of lines of code in
        \vspace{-0.05in}
            \begin{itemize}
                % \item Go
                % \inlineitem Bash
                \item Go
                \inlineitem Java
                \inlineitem Python
                \inlineitem Haskell
                \inlineitem C
                \inlineitem Bash
            \end{itemize}
        \vspace{-0.02in}
            \item over the last few years did thousands of lines of code in
        \vspace{-0.05in}
            \begin{itemize}
                \item JavaScript
                \inlineitem C++
                \inlineitem Clojure / \textsl{ClojureScript}
            \end{itemize}
        \vspace{-0.02in}
            \item used a little over the last few years
        \vspace{-0.05in}
            \begin{itemize}
                \item PHP
                \inlineitem Perl5
                \inlineitem Ruby
                \inlineitem Dart
                \inlineitem \textsl{Pascal}
            \end{itemize}
        \vspace{-0.02in}
            \item heard a lot about (and maybe tried out a little)
        \vspace{-0.05in}
            \begin{itemize}
                \item Rust
                \inlineitem Nim
                \inlineitem Ocaml
                \inlineitem Scala (\textsl{with \textit{Scalaz} library})
                \item Kotlin
                \inlineitem \textsl{Flutter \textit{(Dart)}}
            \end{itemize}
        \end{itemize}
        \item fairly good grasp of GNU/Linux, FreeBSD, OpenBSD operating systems
        \vspace{0.04in}
        \item \textsc{DataBase Management Systems} — over the last few years, very basic experience wth:
        \vspace{-0.07in}
        \renewcommand{\labelitemii}{•}
        \begin{itemize}
            \item MySQL \textsl{(less than three years ago)}
            \inlineitem Redis
            \inlineitem MongoDB
            (\textit{esp. replication configuration})
        \end{itemize}
        \item basic grasp of Docker platform \textsl{(somewhat outdated)}
        \vspace{0.04in}
        \item \textsc{Version Control Systems} \hspace{0.2in}
        \vspace{-0.07in}
        \begin{itemize}
            \item Git (\textsl{good familiarity})
            \inlineitem Mercurial (\textsl{basic grasp})
            \inlineitem Pijul (\textsl{did readup\&experimenting})
        \end{itemize}
        \item an interest in Test Driven Development
        (\textsl{although no proper TDD experience})
        \vspace{0.04in}
        \item professional experience in agile methodologies, comprehension of Scrum
        \item acquaintance with the Nonviolent Communication (NVC) approach
    \end{itemize}
    \vspace{-0.16in}
    \section{Activities \& societes}
        \vspace{0.14in}
    \begin{itemize}
        \item a regular attendee of IT and Agile meetups since 2016 until 2019
        \vspace{-0.03in}
        \item a member of Ingress players' community (ENL faction) since 2013
        \vspace{-0.03in}
        \item an active member of Mensa Poland Society since 2014
        \vspace{-0.01in}
        \item \footnotesize betw. 2016 and 2018, a member of {Scientific Circle of Computer Scientists (\textsl{KNI})
        of NCU (\textsl{UMK}) in Toruń},
        \vspace{-0.04in}
        \begin{itemize}
        \vspace{-0.01in}
            \item conducting Git version control system introductory workshops
        \vspace{-0.03in}
            \item helping organize Clojure introductory workshops
        \end{itemize}
    \end{itemize}
\end{resume}
\vspace{-0.15in}
\begin{center}
    \noindent\makebox[\dimexpr\linewidth]{\rule{\dimexpr\paperwidth-3in}{0.4pt}}
    \footnotesize
    Zgadzam się i wyrażam zgodę bla bla bla tu wstawiać później stopkę wedle życzenia pracodawcy.\\
    A do publikacji na stronie wyrzucić to stąd.
\end{center}
\end{document}